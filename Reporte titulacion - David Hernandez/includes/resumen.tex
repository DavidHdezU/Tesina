\addcontentsline{toc}{chapter}{Resumen}

\begin{center}
\large{\bf Desarrollo de un Tablero de Datos para un Sistema de Información Hospitalaria}

\normalsize{Tesis de Licenciatura}\\
\vspace{0.5cm}

\normalsize{\bf David Hernández Uriostegui\\
Licenciatura en Ciencias de la Computación}\\
Universidad Nacional Autónoma de México

\vspace{1.0cm}

\large{\textbf{Resumen\\}}

\end{center}
La Secretaría de Salud de la Ciudad de México (SEDESA) tiene implementada desde el 2014 la plataforma del Sistema de Administración Médica e Información Hospitalaria (SAMIH), catalogada como una de las mejores al cumplir con el protocolo \text{HL7} (Health Level Seven), en 30 Hospitales de la red hospitalaria institucional. Este sistema de información apoya las actividades en los niveles operativo, táctico y estratégico en los hospitales. Para ello utiliza sistemas que permiten recabar, almacenar y procesar información clínica y administrativa, y los expedientes clínicos de cada uno de los usuarios atendidos, que contienen información relevante para los profesionales de la salud y la historia médica del paciente, sus antecedentes, diagnósticos y pronósticos en salud (Morales-Velázquez, 2019). Mediante SAMIH el sistema de salud de la Ciudad de México puede disponer de información y generar datos de salud útiles para mejorar la gestión médico-administrativa y facilitar la toma de decisiones. Sin embargo, el propósito del SAMIH busca otros fines y para atender necesidades de información como lo fue requerido durante la pandemia del COVID-19 es necesario robustecer ciertas funcionalidades y aumentar otras.\\

Una de las principales necesidades de la SEDESA ante la contingencia sanitaria fue el contar con información de calidad y oportuna, que permita la óptima administración de los recursos en salud y garantizar el derecho efectivo a la salud de la población de la Ciudad de México. Motivo por el cual, urge superar algunas de las limitantes actuales para lograr, en un corto plazo, hacer frente a la pandemia de COVID-19 y, en un mayor plazo, lograr establecer un sistema de información en salud resiliente ante contingencias futuras.\\

Este proyecto busca que la SEDESA y el Gobierno de la Ciudad de México puedan acceder a información contenida en SAMIH mediante un tablero analítico con información estadística de los pacientes que se encuentran hospitalizados. El sistema propuesto permitirá robustecer las funcionalidades de sistema de información hospitalaria de la SEDESA, facilitar la administración de los recursos en salud, mejorar la atención hospitalaria de los pacientes y, en última instancia, contribuir a garantizar el derecho efectivo a la salud de la población de la Ciudad de México (Ayaad et al., 2019).\\

Se publicarán los resultados obtenidos del modelo del tablero analítico y los resultados de la extracción de información de los expedientes clínicos a partir de técnicas de procesamiento de lenguaje natural, contribuyendo a la generación de conocimiento científico y la formación de recursos humanos. Además, el sistema permitirá la extracción sistemática de información relevante de los pacientes hospitalizados para fines de investigación, cuyos resultados, a la vez, contribuirán a la toma de decisiones.
 
%%% Local Variables: 
%%% mode: latex
%%% TeX-master: "tesis"
%%% End: 
